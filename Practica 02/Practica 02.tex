\documentclass[12pt]{article}

\usepackage[all]{xy}
\usepackage{amssymb}
\usepackage{amsmath}
\usepackage{graphics}
\usepackage{enumerate}
\usepackage[utf8]{inputenc}
\usepackage{amsfonts}
\usepackage{makeidx}
\usepackage{graphicx}
\usepackage{synttree}


\usepackage{wasysym}
\begin{document}

\begin{center}

\vspace*{\baselineskip}

{
\bf\fontsize{19}{0}{\selectfont{UNIVERSIDAD NACIONAL  SAN CRISTOBAL DE HUAMANGA}}\\[0.5cm]
\fontsize{11}{0}{FACULTAD DE INGENIERÍA DE MINAS, GEOLOGÍA Y CIVIL}
}

\vspace*{0.5\baselineskip}


{
\bf\fontsize{11}{0}{\selectfont{ESCUELA DE CIENCIAS FISICO Y MATEMATICAS}}
}


\vspace*{\baselineskip}
\includegraphics[width=4.5cm,height=6.5cm]{../unsch.jpg} 
\vspace*{3\baselineskip}

{
\bf\fontsize{13}{0}{\selectfont{Cálculo de Probabilidades}}
}

\vspace*{3\baselineskip}
\end{center}

{
\vfill\bf\fontsize{14}{0}{\selectfont{Docente: Jackson Macoy Romero Plasencia}}\\[0.01cm]
\vfill\bf\fontsize{14}{0}{\selectfont{Alumno: Romeld Jheffry	Vilca Torres}}
}
\begin{center}


{Ayacucho,Perú}\\ 
 2019

\thispagestyle{empty}
\end{center}

\title{\textbf{Práctica 02}}
\date{}
\maketitle

\begin{enumerate}

\item Demostrar las siguientes propiedades\\
\begin{enumerate}[a)]

\item $0\leqslant P(A) \leqslant 1$

\item $P(A^c) = 1- P(A)$\\
Demostración\\
Recordar $A \cup A^c = \Omega \Rightarrow P(\Omega)=1$\\
$P(A \cup A^c)= P(\Omega)$\\
$P(A)+P(A^c)=1$\\
Por lo tanto $P(A^c)= 1-P(A)$

\item $P(A \cup B)= P(A)+P(B)-P(A \cap B); A\cap B \neq \emptyset$\\
Demostración\\
Tenemos:\\
$P(B)=P(B\cap A^c)+P(A\cap B)$\\
$P(A)=P(A\cap B^c)+P(A\cap B)$\\
$ \Rightarrow {P(A)+P(B)}=P(A\cap B^c)+P(A\cap B)+P(B\cap A^c)+P(A\cap B)$\\
$ \Rightarrow P(A)+P(B)-P(A\cap B)=P(A\cap B^c)+P(A\cap B)+P(B\cap A^c) $\\
Por lo tanto:\\
$P(A)+P(B)-P(A\cap B)=P(A\cup B)$

\item $P(A \cup B)= P(A)+P(B); A\cap B = \emptyset $

\item  $P(\displaystyle\bigcup_{i=1}^{n}{A_i})= \displaystyle\sum_{i=1}^{n}P(A_{i})-\displaystyle\sum_{i<1}^{n} P(A_i\cap A_j)+ \displaystyle\sum_{i<j<k}^{n} P(A_i\cap a_j\cap A_k)-...+(-1)^n P(\displaystyle \bigcap_{i=1}^{n}A_i)$

\end{enumerate}
 

\item Dos personas A y B se distribuyen al azar en tres oficinas numerada 1, 2 y 3. Si las dos personas pueden estar en la misma oficina, defina un espacio muestral adecuado.\\
Solución\\
persona 1: A\\
persona 2: B\\
Oficinas: 1,2,3\\


$\Omega =\lbrace (A_1,B_1);(A_2,B_2);(A_3,B_3);$
$(A_1,B_2);(A_1,B_3);(A_2,B_1);(A_2,B_2);$
$(A_3,B_1);(A_3,B_2) \rbrace$


\item Durante el día, una máquina produce tres artículos cuya calidad individual, definida como defectuoso o no defectuoso, se determina al final del día. Describa el espacio muestral generado por la producción diaria.\\
Solución\\

D:Defectuoso\\
B:No defectuoso\\

$\Omega =\lbrace(DDD);(DDB);(DBD);(BDD);(BBD);(BDB);(DBB;(BBB)\rbrace$


\item De un grupo de transistores producidos bajo condiciones similares, se escoge una sola unidad, se coloca bajo prueba en un ambiente similar a su uso diseñado y luego se prueba hasta que falla. Describir el espacio - muestral\\
Solución:\\
transitor elegido:$ X_i; i=1,n$\\
t; tiempo de vida del transitor x:\\
$\Omega = \lbrace 0\leqslant t \leqslant Tesperando \rbrace  $


\item Una urna contiene cuatro fichas numeradas: 2,4,6, y 8 ; una segunda urna contiene cinco fichas numeradas: 1,3,5,7, y 9. Sea un experimento aleatorio que consiste en extraer una ficha de la primera urna y luego una ficha de la segunda urna, describir el espacio muestral.\\
Solución\\
$\Omega$ = $\left\lbrace (x,y)/x=\left\lbrace 2,4,6,8\right\rbrace; y = \left\lbrace 1,3,5,7,9\right\rbrace\right\rbrace$

\item Una urna contiene tres fichas numeradas: 1,2,3; un experimento consiste en lanzar un dado y luego extraer una ficha de la urna. Describir el espacio muestral.\\
Solución\\
$\ U_{1}=\lbrace1,2,3\rbrace       \,   D_{1}=\lbrace1,2,3,4,5,6\rbrace$

A: Lanzar un dado y luego extraer una ficha de la urna 

$\Omega_{A}=\lbrace (x,y) \diagup x \in \lbrace1,2,3,4,5,6\rbrace ;  y \in \lbrace1,2,3\rbrace \rbrace$


\item Una línea de producción clasifica sus productos en defectuosos "D" y no defectuosos "N". De un almacén donde guardan la producción diaria de esta línea, se extraen artículos hasta observar tres defectuosos consecutivos o hasta que se hayan verificado cinco artículos. Construir el espacio -muestral.\\
Solución\\
$\Omega=\lbrace DDD,DDNDD,DDNDN,DDNND,DDNNN,DNDDD,DNDDN,$

$ DNDND,DNDNN,DNNDD,DNNDN,DNNND,DNNNN,NDDD,$

$ NDDND,NDDNN,NDNDD,NDNDN,NDNND,NDNNN,NNDDD,$

$ NNDDN,NNDND,NNDNN,NNNDD,NNNDN,NNNND,NNNNN\rbrace$

\item Lanzar un dado hasta que ocurra el número 4. Hallar el espacio muestral asociado a este experimento.\\
Solución\\
$A = 4$\\
$B\not= 4$\\
$\Omega \lbrace A,BA,BBA,BBBA,...\rbrace$ 

\item Una moneda se lanza tres veces. Describa los siguientes eventos:\\
A = "ocurre por lo menos 2 caras".\\
$\ A=\lbrace CCS,CSC,SCC,CCC\rbrace$
B = "ocurre sello en el tercer lanzamiento".\\
$\ B=\lbrace CCS,CSS,SCS,SSS\rbrace$
C = "ocurre a lo más una cara".\\
$\ C=\lbrace SSS,CSS,SCS,SCC\rbrace$
\item En cierto sector de Lima, hay cuatro supermercados (numeradas 1,2,3,4).Seis damas que viven en ese sector seleccionan al azar y en forma independiente, un supermercado para hacer sus compras sin salir de su sector.
\begin{enumerate}[(a)]
\item Dar un espacio muestral adecuado para este experimento.\\
$\ DAMAS=\lbrace1,2,3,4,5,6\rbrace$    $\ SUPERMERCADOS=\lbrace1,2,3,4\rbrace$
$\Omega=\lbrace (x,y) \diagup x \in \lbrace1,2,3,4,5,6\rbrace   \,  y \in \lbrace1,2,3,4\rbrace \rbrace $
\item Describir los siguientes eventos:
\begin{enumerate}[A: ]

\item "Todas las damas escogen uno de los tres primeros supermercados" \\
$\ A=\lbrace (1,1),(1,2),(1,3),(1,4),(2,1),(2,2),(2,3),(2,4),(3,1),(3,2),$
$\ (3,3),(3,4),(4,1),(4,2),(4,3),(4,4),(5,1),(5,2),(5,3),(5,4),(6,1),$
$\ (6,2),(6,3),(6,4)\rbrace$

\item "Dos escogen el supermercado N$^{\circ}$2 , dos el supermercado N$^{\circ}$3 y las -
otras dos el N$^{\circ}$4".\\
$\ B=\lbrace (           $                             

\item "Dos escogen el supermercado N$^{\circ}$2 y las otras diferentes supermercados".

\end{enumerate}
\end{enumerate}
A : "Todas las damas escogen uno de los tres primeros supermercados"\\
B : "Dos escogen el supermercado N$^{\circ}$2 , dos el supermercado  N$^{\circ}$3 y las otras dos el N$^{\circ}$4 ".\\
C : "Dos escogen el supermercado N$^{\circ}$2 y las otras diferentes supermercados".
\item Tres máquinas idénticas que funcionan independientemente se mantienen funcionando hasta darle de baja y se anota el tiempo que duran. Suponer que ninguno dura más de 10 años.\\
\begin{enumerate}[(a)]
\item Definir un espacio muestral adecuado para este experimento.\\
$\Omega=\lbrace (x,y) \diagup x \in \lbrace1,2,3\rbrace ; y \in \lbrace1,2,3,4,5,6,7,8,9,10\rbrace \rbrace$
\item Describir los siguientes eventos:
\begin{enumerate}[A: ]

\item "Las tres máquinas duran más de 8 años".\\[0.2cm]
$\ A=\lbrace (1,8),(1,9),(1,10),(2,8),(2,9),(2,10),(3,8),$

$ \ (3,9),(3,10)\rbrace$

\item "El menor tiempo de duración de los tres es de 7 años".\\
$\ A=\lbrace (1,7),(1,8),(1,9),(1,10),(2,7),(2,8),(2,9),(2,10),(3,7)$

$\ (3,8),(3,9),(3,10)\rbrace$

\item "El menor tiempo de duración de los tres es de 7 años".\\
$\ A=\lbrace (1,7),(1,8),(1,9),(1,10),(2,7),(2,8),(2,9),(2,10),(3,7)$


$\ (3,8),(3,9),(3,10)\rbrace$
\item El mayor tiempo de duración de los tres es de 9 años".\\
$\ D=\lbrace (x,y) \diagup x \in \lbrace1,2,3\rbrace ; y \in \lbrace1,2,3,4,5,6,7,8,9\rbrace \rbrace$


\end{enumerate}
\end{enumerate}
A : "Las tres máquinas duran más de 8 años".\\
B : "El menor tiempo de duración de los tres es de 7 años".\\
C : "Ninguna es dada de baja antes de los 9 años".\\
D : "El mayor tiempo de duración de los tres es de 9 años".
\item En el espacio muestral del problema 4, describe los siguientes eventos:\\
A : "Ocurre al menos 2 artículos no defectuosos".\\
B : "Ocurre exactamente 2 artículos no defectuosos".\\
\begin{enumerate}[A: ]

\item "Ocurre al menos 2 artículos no defectuosos".\\[0.2cm]

$\ A=\lbrace DNN,NDN,NND,NNN\rbrace$

\item "Ocurre exactamente 2 artículos no defectuosos"\\
$\ A=\lbrace DNN,NDN,NND\rbrace$

\end{enumerate}
\item En el problema 16, describir el evento, "se necesitan por lo menos 5 lanzamientos".\\
Solución\\
Se necesitan por lo menos 5 lanzamientos = $\lbrace xxxx4.xxxxx4,xxxxxx4,....\rbrace$ ; donde x = obtener un número diferente de 4 .
\item El gerente general de una firma comercial, entrevista a 10 aspirantes a
un puesto. Cada uno de los aspirantes es calificado como: Deficiente, Regular, Bueno, Excelente.

\item Considere el experimento de contar el número de carros que pasan por un
punto de una autopista. Describa los siguientes eventos:

\begin{enumerate}[A; ]

\item "Pasan un número par de carros".

$\ A=\lbrace0,2,4,6,8,10,....\rbrace$

\item "El número de carros que pasan es múltiplo de 6 ".

$\ B=\lbrace0,6,12,18,....\rbrace$

\item "Pasan por lo menos 20 carros"

$\ C=\lbrace20,21,22,23,24,...\rbrace$

\item "Pasan a lo más 15 carros".
 
$\ D=\lbrace1,2,3,4,.....,14,15\rbrace$

\end{enumerate}



\end{enumerate}
\end{document}